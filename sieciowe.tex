\documentclass[11pt,a4paper]{article}

% Kodowanie i obsługa języka polskiego
\usepackage[utf8]{inputenc}      % Kodowanie wejścia
\usepackage[T1]{fontenc}         % Kodowanie fontów
\usepackage[polish]{babel}       % Obsługa języka polskiego

% Pakiety matematyczne i inne przydatne
\usepackage{amsmath, amssymb, amsthm}  % Pakiety do matematyki
\usepackage{graphicx}                  % Obsługa grafiki
\usepackage{hyperref}                  % Linki i spis treści
\usepackage{geometry}                  % Ustawienie marginesów
\geometry{margin=2cm}                  % Ustawienie marginesów na 2 cm

% Dodatkowe ustawienia
\hypersetup{colorlinks=true, linkcolor=blue, urlcolor=blue}  % Ustawienia linków
\title{Notatki Technologie Sieciowe}
\author{Jakub Kogut}
\date{\today}

\begin{document}

\maketitle

\tableofcontents  % Spis treści (opcjonalnie)
\newpage

\section{Wstęp}
Notatki z przedmiotu \textit{Technologie sieciowe} na kierunku Informatyka Algorytmiczna na Politechnice Wrocławskiej na semestrze 4 2025r.
\begin{itemize}
    \item Prowadzący: dr inż. Łukasz Krzywiecki
    \item email: \href{mailto:lukasz.krzywiecki@pwr.edu.pl}{mail}
    \item konsultacje D1/210
        \begin{itemize}
            \item Mon. 11:00 - 13:00, MSTeams, D-1:210
            \item Wed. 11:00 - 13:00, MSTeams, D-1:210 (The 1st and the 2nd week of the month)
            \item Fri. 15:00 - 17:00, MSTeams, (Last 2 weeks of the month)
        \end{itemize}
\end{itemize}

\section{Wykład \date{2025-03-11}}
\subsection{Czym jest sieć komputerowa?}
%załącz grafike sieci do pliku jako obraz ./Weighted-Graph.png
\begin{figure}[h]
    \centering
    \includegraphics[width=0.5\textwidth]{./Weighted-Graph.png}
    \caption{Przykład sieci komputerowej}
\end{figure}
Jest to ważony graf, w którym nodami są komputery, a krawędziami połączenia między nimi. Waga krawędzi można interpretować jako:
\begin{itemize}
    \item przepustowość łącza -- ilość danych, które można wysłać przez łącze
    \item opóźnienie
    \item szerokość pasma -- szerokość zakresu częstotliwości, który jest wykorzystywany przez nadawane lub odbierane sygnały w danym medium transmisyjnym. Szerokość pasma jest wyrażana w różnicy pomiędzy najwyższą a najniższą częstotliwością składnika transmitowanego sygnału
\end{itemize}
\subsection{Jak wyglądały kiedyś połączenia sieciowe?}
Używano kabla koncetrycznego \texiit{Coaxial Cable} (10Mb/s) oraz kabla skrętkowego \textit{Twisted Pair} (około 100Mb/s). Współcześnie używa się światłowodów, które mają przepustowość rzędu 10Gb/s.
\subsection{Jak dane są przesyłane w kablu?}
Istnieją różne sposoby modulacji sygnału:
\begin{itemize}
    \item częstotliwościowa -- polegajaca na zmianie częstotliwości sygnału
    \item amplitudowa -- polegająca na zmianie amplitudy sygnału
    \item fazowa -- polegająca na zmianie fazy sygnału
    \item jeżeli używamy światłowodu, to używamy modulacji światła
    \item mieszane:
        \begin{itemize}
            \item QAM -- polegająca na zmianie amplitudy i fazy sygnału, użwyają konstelacji czętotliwości
        \end{itemize}
\end{itemize}
\subsection{Jak wygląda komunikacja w protokole Ethernet?}
Każda wiadomość przesyłana przez sieć opakowana jest w \textbf{ramkę Ethernetową}. Wkładane są w nią kolejne warstwy odpowiednich protokołów.
\begin{enumerate}
    \item Ethernet
    \item IP -- Internet Protocol. Na jego warstwie znajduje się protokół ICMP używany przez program \textit{Ping}.
    \item TCP/UDP -- Transmiton Control Protocol/User Datagram Protocol
\end{enumerate}

\section{\date{2025-04-31}}
Kilka słów o 2. liscie na laboratoria: jej termin został przesunięty o 2 tygodnie oraz aby przyspieszyć sprawdzanie mamy 2 zadania z niej wykonać w 3 osobowych grupach.
\subsection{Sieci Lokalne}
W sieciach lokalnych występują generalnie jeden kanał komuniakcyjny, który jest współdzielony przez wszystkie urządzenia. Stacje bazowe ``rywalizują'' o przesył/dostęp do kanału. Jest to tak zwany \textbf{Kanał typu rozsiewczego}. Jednakże, w takich topologiach występują zagłuszenia, co wymuszało stosowanie ``\textit{wzmacniaków}'' sygnału. Są to urządzenia analogowe, które wzmacniają sygnał, ale również szum. Współcześnie stosuje się \textbf{przełączniki} \textit{Switch}, które są urządzeniami cyfrowymi, które przesyłają dane tylko do odpowiednich stacji, co eliminuje zagłuszenia. Standardy określają ile może być \textbf{segmentów} w sieci lokalnej. Każda maszyna w takiej sieci jest równie ważna.
W tego typu topologiach występują kolizje powodowane jednoczesnym przesyłem sygnału. Jeśli odbiornik znajduje się w miejscu interferencji dwóch fal to odbierze sygnał zniekształcony, szum.
\subsection{Intra--sieć}
Chodzi tu o \textit{internet}. Jest ona odpowiedzialna za \textbf{trasowanie} (\textit{routing}), również za abstrakcje warstwy sieciowej, aby umożliwić niezależność sieci lokalnych podłączonych do chmury.\\
Jak reprezentować sieć? Robimy to za pomocą grafu, w którym wierzchołkami są routery, a krawędziami łącza między nimi. Wierzchołki te są połączone z innymi wierzchołkami, które reprezentują inne sieci. Wierzchołki te nazywamy \textbf{bramkami} (\textit{gateways}). Edge w grafie reprezentuje łącze między dwoma sieciami, są to tzw \textbf{duplexy}, ponieważ komunikacja zachodzi w obie strony. Krawędzie możemy charakteryzować liczbami -- \textbf{wagami}, są to pewne funkcje oznaczające np. przepustowość łącza.
\subsubsection{Chararakterystyki sieci}
\begin{itemize}
    \item \textbf{Przepustowość} -- ilość danych, którą można przesłać w jednostce czasu. Oznaczana przez $c$.
    \item \textbf{Niezawodność} -- prawdopodobieństwo, że sieć działa poprawnie. Oznaczana przez $p$. Wartość $1-p$ to prawdopodobieństwo, że sieć nie działa.
    \item \textbf{Opóźnienie} -- czas, jaki upływa od wysłania pakietu do jego odbioru
    \item \textbf{Szerokość pasma} -- szerokość zakresu częstotliwości, który jest wykorzystywany przez nadawane lub odbierane sygnały w danym medium transmisyjnym. Szerokość pasma jest wyrażana w różnicy pomiędzy najwyższą a najniższą częstotliwością składnika transmitowanego sygnału
\end{itemize}

\section{ważne}
Ważna jest niezawodność w przesyłaniu tych bitów
\begin{itemize}
    \item najniżej jest sygnał, fale itp.
    \item dalej mamy ramkę ethernetową. Już na tym etapie możemy sprawdzić, czy ramka jest poprawna na podstawie sumy kontrolnej (algorytm CRC).
    \item do ramek wkładamy inne paczki, które nazywamy datagramami/pakietami, które należą do protokołu sieciowego ip4 lub ip6. Tutaj rutery zajmują się wyznaczaniem trasy -- routing/trasowanie -- czyli wybieraniem najkrótszej trasy do danego hosta. Dane potrzebne do routowania są zaszyte w nagłówku pakietu.\footnote{routery aktualizuja sobie \textbf{tablice routingu} -- jest to tablica, która zawiera informacje o tym, jak dotrzeć do danego hosta.} Tutaj również można przeprowadzić werfikację poprawności pakietu. Jeżeli pakiet jest niepoprawny, to jest on odrzucany. W przeciwnym wypadku jest przekazywany do kolejnej warstwy.
    \item Następnie mamy warstwę transportową, która zajmuje się przesyłaniem danych między hostami. W tej warstwie mamy dwa protokoły: TCP i UDP. TCP jest protokołem połączeniowym, który zapewnia niezawodność przesyłania danych, natomiast UDP jest protokołem bezpołączeniowym, który nie zapewnia niezawodności przesyłania danych. W tej warstwie również możemy przeprowadzić werfikację poprawności pakietu. TCP zapewnia niezawodność przesyłania danych poprzez mechanizm potwierdzeń i retransmisji. UDP nie zapewnia niezawodności, ale jest szybszy, ponieważ nie wymaga potwierdzeń.

        \textbf{WAŻNE!} TCP zapewnia niezawodność poprzez \textbf{pozytywne potwierdzenie} i \textbf{retransmisje}. Również zapewnia \textbf{kontrole przepływu} -- efektywne wykożystanie kanału komunikacyjnego. W przypadku TCP mamy \textbf{okno przesuwne} -- jest to mechanizm, który pozwala na przesyłanie wielu pakietów bez oczekiwania na potwierdzenie każdego z nich. Okno przesuwne jest dynamicznie dostosowywane w zależności od przepustowości łącza i opóźnienia. TCP posiada również \textbf{reakcje na przeciążenia}, nadawca kontroluje ilość wysyłanych danych w zależności od stanu sieci. W przypadku przeciążenia sieci, nadawca zmniejsza ilość wysyłanych danych, a w przypadku braku przeciążenia, zwiększa ją.

        Protokoły te zajmóją sie przesyłaniem \textbf{bajtów/oktetów}. Jest to system \textbf{połączeniowy}, musi zostać otwarty kanał komunijaci oraz potwierdzenie nadawcy i odbiorcy. Po skończonym nadawaniu kanał/połączenie musi zostać zamknięte. W przypadku TCP mamy \textbf{trójfazowe uzgadnianie połączenia} -- jest to proces, który pozwala na ustalenie parametrów połączenia między nadawcą a odbiorcą. Proces ten składa się z trzech kroków:
        \begin{enumerate}
            \item Nadawca wysyła do odbiorcy pakiet SYN (synchronizacja), w którym informuje o chęci nawiązania połączenia.
            \item Odbiorca odpowiada pakietem SYN-ACK (synchronizacja i potwierdzenie), w którym informuje nadawcę o chęci nawiązania połączenia oraz potwierdza otrzymanie pakietu SYN.
            \item Nadawca wysyła pakiet ACK (potwierdzenie), w którym potwierdza otrzymanie pakietu SYN-ACK i kończy proces uzgadniania połączenia.
        \end{enumerate}
\begin{table}[ht]
  \centering
  \renewcommand{\arraystretch}{1.3}
  \begin{tabular}{|c|c|c|c|c|c|c|c|c|c|c|c|c|c|c|c|}
    \hline
    \multicolumn{2}{|c|}{\textbf{Source Port (16 bits)}}
      & \multicolumn{2}{c|}{\textbf{Destination Port (16 bits)}} \\ \hline

    \multicolumn{4}{|c|}{\textbf{Sequence Number (32 bits)}} \\ \hline

    \multicolumn{4}{|c|}{\textbf{Acknowledgment Number (32 bits)}} \\ \hline

    \textbf{Data Offset (4 bits)}
      & \textbf{Reserved (6 bits)}
      & \textbf{Flags (6 bits)}
      & \textbf{Window Size (16 bits)} \\ \hline

    \textbf{Checksum (16 bits)}
      & \textbf{Urgent Pointer (16 bits)}
      & \multicolumn{2}{c|}{\textbf{Options \& Padding (variable)}} \\ \hline
  \end{tabular}
  \caption{TCP Header Segment Fields}
  \label{tab:tcp-segment}
\end{table}

\end{itemize}
\section{Podsumowanie}

\end{document}
