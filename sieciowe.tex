\documentclass[11pt,a4paper]{article}

% Kodowanie i obsługa języka polskiego
\usepackage[utf8]{inputenc}      % Kodowanie wejścia
\usepackage[T1]{fontenc}         % Kodowanie fontów
\usepackage[polish]{babel}       % Obsługa języka polskiego

% Pakiety matematyczne i inne przydatne
\usepackage{amsmath, amssymb, amsthm}  % Pakiety do matematyki
\usepackage{graphicx}                  % Obsługa grafiki
\usepackage{hyperref}                  % Linki i spis treści
\usepackage{geometry}                  % Ustawienie marginesów
\geometry{margin=2cm}                  % Ustawienie marginesów na 2 cm

% Dodatkowe ustawienia
\hypersetup{colorlinks=true, linkcolor=blue, urlcolor=blue}  % Ustawienia linków
\title{Notatki Technologie Sieciowe}
\author{Jakub Kogut}
\date{\today}

\begin{document}

\maketitle

\tableofcontents  % Spis treści (opcjonalnie)
\newpage

\section{Wstęp}
Notatki z przedmiotu \textit{Technologie sieciowe} na kierunku Informatyka Algorytmiczna na Politechnice Wrocławskiej na semestrze 4 2025r.
\begin{itemize}
    \item Prowadzący: dr inż. Łukasz Krzywiecki
    \item email: \href{mailto:lukasz.krzywiecki@pwr.edu.pl}{mail}
    \item konsultacje D1/210
        \begin{itemize}
            \item Mon. 11:00 - 13:00, MSTeams, D-1:210
            \item Wed. 11:00 - 13:00, MSTeams, D-1:210 (The 1st and the 2nd week of the month)
            \item Fri. 15:00 - 17:00, MSTeams, (Last 2 weeks of the month)
        \end{itemize}
\end{itemize}

\section{Wykład \date{2025-03-11}}
\subsection{Czym jest sieć komputerowa?}
%załącz grafike sieci do pliku jako obraz ./Weighted-Graph.png
\begin{figure}[h]
    \centering
    \includegraphics[width=0.5\textwidth]{./Weighted-Graph.png}
    \caption{Przykład sieci komputerowej}
\end{figure}
Jest to ważony graf, w którym nodami są komputery, a krawędziami połączenia między nimi. Waga krawędzi można interpretować jako:
\begin{itemize}
    \item przepustowość łącza -- ilość danych, które można wysłać przez łącze
    \item opóźnienie
    \item szerokość pasma -- szerokość zakresu częstotliwości, który jest wykorzystywany przez nadawane lub odbierane sygnały w danym medium transmisyjnym. Szerokość pasma jest wyrażana w różnicy pomiędzy najwyższą a najniższą częstotliwością składnika transmitowanego sygnału
\end{itemize}
\subsection{Jak wyglądały kiedyś połączenia sieciowe?}
Używano kabla koncetrycznego \texiit{Coaxial Cable} (10Mb/s) oraz kabla skrętkowego \textit{Twisted Pair} (około 100Mb/s). Współcześnie używa się światłowodów, które mają przepustowość rzędu 10Gb/s.
\subsection{Jak dane są przesyłane w kablu?}
Istnieją różne sposoby modulacji sygnału:
\begin{itemize}
    \item częstotliwościowa -- polegajaca na zmianie częstotliwości sygnału
    \item amplitudowa -- polegająca na zmianie amplitudy sygnału
    \item fazowa -- polegająca na zmianie fazy sygnału
    \item jeżeli używamy światłowodu, to używamy modulacji światła
    \item mieszane:
        \begin{itemize}
            \item QAM -- polegająca na zmianie amplitudy i fazy sygnału, użwyają konstelacji czętotliwości
        \end{itemize}
\end{itemize}
\subsection{Jak wygląda komunikacja w protokole Ethernet?}
Każda wiadomość przesyłana przez sieć opakowana jest w \textbf{ramkę Ethernetową}. Wkładane są w nią kolejne warstwy odpowiednich protokołów.
\begin{enumerate}
    \item Ethernet
    \item IP -- Internet Protocol. Na jego warstwie znajduje się protokół ICMP używany przez program \textit{Ping}.
    \item TCP/UDP -- Transmiton Control Protocol/User Datagram Protocol
\end{enumerate}

\section{Podsumowanie}

\end{document}
